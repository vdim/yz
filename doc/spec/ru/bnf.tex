Приведем формальное описание синтаксиса языка в нотации Бэкуса-Наура, %TODO: ссылку на форму
которая широко применяется для описания контекстно-свободных грамматик или BNF 
(Backus-Naur From --- форма Бэкуса-Наура). Жирным шрифтом выделены {\bf терминалы}, 
курсивом --- {\it нетерминалы}. 

{\it query} $\to$ {\it select\_block} {\it other} | {\it function\_block}

{\it select\_block} $\to$ {\bf s[} {\it what} {\bf ]}

{\it other} $\to$ $\varepsilon$ | {\it where\_block} | {\it where\_block order\_block}
    | {\it order\_block} | {\it order\_block where\_block}
    | {\it tuning\_block} | {\it where\_block tuning\_block} | {\it tuning\_block where\_block} 
    | {\it where\_block order\_block tuning\_block} | {\it where\_block tuning\_block order\_block} 
    | {\it order\_block where\_block tuning\_block} | {\it order\_block tuning\_block where\_block}
    | {\it tuning\_block where\_block order\_block} | {\it tuning\_block order\_block where\_block};

{\it what} $\to$ {\it id} | {\it what} {\bf ,} {\it id} | {\it what} {\bf (} {\it what} {\bf )}

{\it where\_block} $\to$ {\bf w[} {\it expr} {\bf ]}

{\it expr} $\to$ {\it expr and term} | {\it term}

{\it and} $\to$ {\bf and} | {\bf \&\&}  
