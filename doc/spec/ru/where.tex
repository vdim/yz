Язык позволяет задавать ограничения на элементы кортежа прямо в определении структуры кортежа.
Например, запрос
\query{room.number=``215''}
вернет список комнат, номером которых является значение ``215''.

Для организации сложных условий может использоваться: 
    \begin{itemize}\addtolength{\itemsep}{-0.7\baselineskip}
	\item логическое ``И'' (знак ``$\&\&$'' или слово {\itshape{\bfseries and}}),
	\item логическое ``ИЛИ'' (знак ``$||$'' или слово {\itshape{\bfseries or}}),
	\item отрицание (знак ``!''),
	\item круглые скобки для группировки сложных условий.
    \end{itemize}

В ограничениях могут использоваться следующие бинарные операции:
    \begin{itemize}\addtolength{\itemsep}{-0.7\baselineskip}
	\item ``='' --- равенство. 
%		Применимы к любым типам данных. 
		Обозначает одну и ту же величину 
		(т.е. 1 будет равно 1.0) или один и тот же объект.
	\item ``>'', ``>='', ``<'', ``<='' --- знаки категории ``больше-меньше''. 
%		Применимы к числовым и строковым типам.
	\item ``\symbol{"7E}'' --- знак регулярного выражения. 
%		Применим к строковым типам данных.
    \end{itemize}

Набор типов, к которым могут применяться бинарные операции, задается реализацией
данной спецификации. Однако, реализация должна поддерживать, по-крайней мере,
следующие представления данных:
    \begin{itemize}\addtolength{\itemsep}{-0.7\baselineskip}
	\item Строка.
	    \query{building.name=``ГК''}
	    Строка должна быть заключена в двойные кавычки. Внутри двойных кавычек может
	    использоваться любой символ. Чтобы использовать двойную кавычку, необходимо
	    ее экранировать с помощью обратного слеша (\symbol{"5C}), сам обратный слеш также
	    необходимо экранировать:
	    \query{building.name=``Building \symbol{"5C}``Main\symbol{"5C}''''}

	\item Число. Числа могут быть положительными, отрицательными, целыми и с плавающей точкой.
	    \query{floor.number>=1}

	\item Элемент запроса. Например, в запросе
	    \query{floor.number>=1} левая часть бинарной операции (\cl{floor.number}) является элементом.
	
	\item Ключевые слова \cl{true} и \cl{false} для возможности сравнения данных логического типа:
	    \queryexpl{Найти все маршрутизаторы сети}{device.forwarding=true}

	\item Ключевое слово \cl{nil} для обозначение элемента как не содержащего никакого значения:
	    \queryexpl{Найти все здания с заданным значением свойства \cl{name}}{building.name!=nil}

   \end{itemize}

Если типы правой и левой частей не совпадает, то пользователь должен получить сообщение
об ошибке. Например, в запросе
\query{floor.number=``IMO''}
поле \cl{number} класса \cl{Floor} имеет тип целое число. Сравнивая его со строкой,
пользователь должен получить сообщение об ошибке.






\subsubsection{Сложные ограничения}
ЯЗ позволяет задавать сложные ограничения как на одно свойство класса, так и на множество.
%Сложные условия можно использовать на одно свойство класса, 
Например в запросе
\queryexpl{Найти комнаты с номерами либо ``215'', либо ``217''}{room.number=(``215'' ``217'')}
указано несколько ограничений на одно свойство \cl{number}. Подразумевается, что эти значения
находятся в соотношении ``ИЛИ'', потому что при знаке $=$ логический оператор  ``И''
превращает данный запрос в бессмыслицу. Также логический оператор можно указать явно:
\query{room.number=(``215''~||~``217'')}


В предыдущем запросе знак ``='' можно переопределить следующим образом:
\queryexpl{Выбрать комнаты с номерами либо 215, либо начинающиеся с цифры 3}{room.number=(``215'' || \symbol{"7E} ``3.*'')}
Такая возможность может быть полезна, если необходимо ограничить значение свойства,
используя разные знаки соотношения:
\queryexpl{Выбрать этажи с первого по пятый}{floor.number(>=1~$\&\&$~<=5)}
Обратите внимание, что в том случае
если все знаки определены внутри скобок, указание знака перед скобками необязательно.

Если необходимо ограничение на несколько свойств одного объекта, то задавать его
нужно в скобках после точки. Например:
\queryexpl{Найти комнату с номером ``215'' и именем ``IMO\_215''}{room.(number=``215'' $\&\&$ name=``IMO\_215'')}

Приведем дополнительные примеры, демонстрирующие возможность определения сложных условия:
\query{
    room.(!number=``215'' $\&\&$ number \symbol{"7E} ``2.*'')
    \vspace{0.5cm}

    room.number(!=``215'' $\&\&$ \symbol{"7E} ``2.*'')
    \vspace{0.5cm}
    
    floor.number((>= 1 $\&\&$ <=5) || =10)
    \vspace{0.5cm}
	
    room.(number=``215'' \\
    \makebox[3em][r]{} $\&\&$ name=(``IMO\_215'' ``PMIK\_215'') \\
    \makebox[3em][r]{} $\&\&$ description=``Main room of the IMO'')
    \vspace{0.5cm}
    
    room.(number=``215'' \\
    \makebox[3em][r]{} $||$ (floor.number=2\\
    \makebox[3em][r]{} $\&\&$ description=``Main room of the IMO''))

}







\subsubsection{Вложенные обращения}
Возможно, что потребуется выбрать некоторое подмножество вложенного обращения. Для этого
это подмножество необходимо заключить в скобки:
\query{(room.floor).building.name=``ГК''}
Вышеприведенный запрос вернет список этажей в здании(\cl{building}) с именем ``ГК''. 
Обратите внимание, что результат этого запроса, может отличаться от результата запроса
\query{(floor).building.name=``ГК''}
В результат второго запроса войдут все этажи здания с именем ``ГК'', а в первый только те,
в которых имеются комнаты. Второй запрос будет эквивалентен запросу
\query{floor.building.name=``ГК''},
так как без указания скобок во вложенных обращениях подразумевается, что выбирается первый указанный 
элемент. Такое соглашение действует только при наличии условия, так как запрос 
\query{floor.building.name}
вернет имена зданий, в которых есть этажи, но запрос
\query{(floor.building).name}
уже вернет не имена, а сами объекты \cl{building}.

Во вложенных обращения также позволяется задавать ограничения на свойства. Например,
в запросе
\query{floor.(number(>1 $\&\&$ <4)).room}
будут выбраны комнаты на этажах с 1 по 4.

Еще примеры:
\queryexpl{Найти устойства на каждом из этажей, номер которого равен 1 и 
    на котором есть комната с номером ``215'' и именем ``IMO''}
    {room.(name=``IMO'' $\&\&$ number=``215'').floor.(number=1).device}
%\queryexpl{Предыдущий запрос эквивалентен следующему}
%    {device.(floor.number=1 $\&\&$ room.(name=``IMO'' $\&\&$ number=``2''))}
\queryexpl{Найти комнаты в здании главного корпуса}
    {building.(name=``ГК'').room}
\queryexpl{Предыдущий запрос эквивалентен следующему}
    {room.building.name=``ГК''}



\subsubsection{Выборка свойств}
Если необходимо выбрать не сам объект, а его свойство, но при этом задать ограничение на другое
свойство, то сделать это можно, указав нужное свойство в квадратных скобках:
\queryexpl{Выбрать имена комнат с номером ``215''}{room[name].number=``215''}

Однако, запись запроса ``выбрать номера комнат, начинающихся с 2'' выглядела бы
избыточной:
\query{room[number].number \symbol{"7E} ``2.*''},
поэтому ЯЗ позволяет задавать ограничение прямо в квадратных скобках:
\query{room[number \symbol{"7E} ``2.*'']}.

На одно свойство распространяются все правила, описанные выше. Примеры
\query{
    room[number=(``215'' ``217'')]
    \vspace{0.5cm}

    floor[number(>=1 $\&\&$ <=5)]
    \vspace{0.5cm}

    floor[number((>=1 $\&\&$ <=5) || =10)]
}


Однако, если в квадратных скобках указаны два или более свойства с условиями, то между 
ними подразумевается логические условие ``И'', т.е такой запрос
\query{room[number=``215'' name=``IMO215'']}
можно перевести на естественный язык как ``найти все комнаты с номером 215 и именем IMO215, 
при этом в результат включить только номера и имена этих комнат''.
Если необходимо задавать сложные условия по нескольким свойствам одного объекта, то тогда их необходимо вынести:
\query{room[number name].(number=``215'' || name=``IMO215'')}






\subsubsection{Связанные элементы}
Условие на элемент кортежа подразумевает собой следующее действие: каждый кортеж из полученного набора кортежей
проверяется на заданное ограничение. Таким образом запрос
\query{building (room.number=``215'')}
вернет все здания, а у здания только те комнаты, номером которых является ``215''. Если в здании нет комнаты с 
номером ``215'', то в этом случае список комнат у такого здания будет пуст, но само здание будет
включено в результат запроса. Для того чтобы получить
только те здания, которые имеют комнату с номером ``215'', а также указать список комнат этих зданий, необходимо
выполнить запрос:
\query{building.room.number=``215'' (room)}
Обратите внимание, что список комнат будет полным, там будут не только комнаты с номером ``215'' (однако этот список будет 
обязательно содержать комнату с номером ``215''). Чтобы выбрать здания, имеющие комнату с номером ``215'', и 
указать только такие комнаты, необходимо выполнить запрос:
\query{building.room.number=``215'' (room.number=``215'')}

Стоит помнить про то, что элемент на следующем уровне иерархии привязывается к 
последнему выбираемому элементу на предыдущем уровне иерархии. Поэтому
\queryexpl{\cl{floor} привязывается к \cl{building}}{building.device.room.number=``215'' (floor)}
\queryexpl{\cl{floor} привязывается к \cl{device}}{(building.device).room.number=``215'' (floor)}






\subsubsection{Рекурсивные ссылки}
К запросам с рекурсивными ссылками применимы все обозначенные выше правила.
Например, запрос
\query{simpleou.(name=``Кафедра ИМО'').parent+}
выдаст всю иерархию \cl{parent} только для тех \cl{simpleou}, имена которых равны значению ``Кафедра ИМО''.

Если задать ограничение на саму рекурсивную ссылку, то данное ограничение будет применяться к каждому
объекту из иерархии. В результат такого запроса попадут только те иерархии, в которых есть хотя бы
один объект удовлетворяющий ограничению.
%Также можно задать ограничение на саму рекурсивную ссылку:
Например, 
\query{simpleou.parent+.name=``МатФак''}
Данный запрос можно трактовать как для каждого простой организационной единицы получить 
все вышестоящий подразделения, имя одного из которых равно значению ``МатФак''. Результат
такого запроса на примере модели, изображенной на рис.~\ref{fig:model-snapshot-org}, 
будет выглядеть следующим образом:

\begin{center}
    \begin{tabular}{|l|l|}
	\hline
	\it{parent} & \it{parent}\\[5pt]
	\hline
	\hline
	МатФак & ПетрГУ \\
	\hline
    \end{tabular}
\end{center}


