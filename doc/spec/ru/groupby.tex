\begin{center}
\end{center}
%Одной из основных особенностей ЯЗ является то, что он позволяет группировать
%связанные между собой объекты (см. подробнее~\ref{tied_queries}).
Часто бывает необходимым передать в функцию в качестве параметра список объектов сгруппированных
относительно другого набора объектов. Количество вызовов функции в этом случае будет соответствовать
количеству объектов второго набора.
%обрабатывать объекты сгруппировав их относительно других объектов. 
Часть таких возможностей обеспечивается вызовом функции на не первом уровне 
иерархии запроса (см. подробнее в~\ref{func-param}). Например, в запросе
\query{building (@count(room))}
список объектов \cl{room} формируется относительно объекта \cl{building},
находящегося на более высоком уровне иерархии. Таким образом мы можем группировать
списки, которые передаются в функцию.

Однако, идеология такого подхода несколько отличается от группировки, так как 
подразумевает под собой связь между объектами модели, поэтому этот подход
не позволяет, во-первых, не включать в результат объект более
высокого уровня иерархии, а во вторых группировать не относительно объекта, а 
относительно свойства объекта. Чтобы решить эти проблемы в ЯЗ был
введен специальный модификатор ``:'', который разделяет имя класса, объекты которого будут группироваться,
и определение класса или свойства класса, относительно которого будет происходить группировка.

%Вначале рассмотрим группировку на уровне запроса, а затем
Рассмотрим следующий пример:
\query{floor:number}
Результат данного запроса будет состоять из кортежей, элементом которых будет
список объектов класса \cl{floor}. У всех объектов \cl{floor} в списке будет
один и тот же номер, так как запрос подразумевает то, что объекты \cl{floor}
должны быть сгруппированы по поле \cl{number}.

Если в запросе указывается необходимость связать элемент со списком сгруппированных объектов,
то в набор связанных объектов попадут все объекты, которые связаны с объектами из списка
сгруппированных объектов.
Например, в запросе
\query{floor:number (@count(room))}
объекты класса \cl{room} будут связываться с объектами класса \cl{floor}, сгруппированными относительно свойства
\cl{number}. Таким образом, с точки зрения модели в функцию \cl{count} будет передаваться набор комнат, находящихся
на этажах с одинаковыми номерами.

Группировать можно как по свойству класса, так и по связанному с классом другому классу.
Например, результатом запроса
\query{room:building}
будут списки комнат (\cl{room}), находящихся в одном и том же здании (\cl{building}).
Приоритет поиска отношений между группируемым классом и определенем группировки
точно такой же как и при поиске связей (подробнее см. в~\ref{simple-select}).

