Для того чтобы позволить пользователю получать в качестве результата запроса уже
преобразованную в требуемый вид выборку из БД, в язык было введено понятие функции.
Возможности, которые есть у пользователя для реализации функции определяются
реализацией данной спецификации. Например, в ЯЗ4J функцию
можно реализовывать либо на чистом Java, объявляя ее как статический метод некоторого класса, 
либо на языке, который предоставляет возможность вызвать выполнение реализованного на нем
кода через Scripting API.%~\cite{Bosanac:2007:SJL:1406211}. 

В качестве одного из элементов результата запроса может быть использован
результат, который возвращает функция. Вызов функции выглядит следующим образом
\query{@funcn}

Функция может быть:
    \begin{itemize}\addtolength{\itemsep}{-0.7\baselineskip}
	\item частью структуры результата запроса\footnote{Здесь и далее для примеров используется функция
\cl{count}, которая в качестве аргумента принимает список, а возвращает количество элементов в списке.
Подробнее смотри в~\ref{std-funcs}}:
	    \queryexpl{Найти здания и подсчитать количество этажей в них}{building (@count(room))}
	\item запросом:
	    \query{@count(room)}
	\item вызвана в условии:
	    \queryexpl{Найти здания с количеством этажей больше 3}{building.(@count(room)>3)}
	\item параметром функции:
	    \query{@funcn(1, @anotherfuncn())}
    \end{itemize}





\subsubsection{Параметры}\label{func-param}
Параметры функции указываются в круглых скобках после имени функции через точку с запятой
\footnote{Использование точки с запятой, отличного от большинства языков программирования знака, 
использующих запятую, связано с тем, что одним из видом параметра может быть запрос на ЯЗ, 
в синтаксисе которого используется запятая.}:
\query{@funcn(par1; par2; par3)}
Параметром функции может быть: 
    \begin{itemize}\addtolength{\itemsep}{-0.7\baselineskip}
	\item строка, заключенная в двойные кавычки;
	\item число;
	\item логическое ``да'' (ключевое слово \cl{true}) или ``нет'' (ключевое слово \cl{false});
	\item ключевое слово \cl{nil}, обозначающее пустое значени;
	\item запрос.
%	\item результат вызова другой функции;
%	\item объект предметной области;
%	\item список объектов предметной области.
    \end{itemize}

В том случае если параметром функции является запрос, то в качестве значения
параметра будет передан результат данного запроса, т.е. либо элемент, либо
список кортежей. Например, в запросе
\queryexpl{Подсчитать количество комнат}{@count(room)}
параметром функции \cl{@count} является запрос 
\query{room}
Результатом его выполнения является список комнат (\cl{room}), который и будет
передан функции.

Если необходимо, чтобы функция вызывалась с каждым кортежем отдельно из списка
результата запроса, то для этого необходимо в начале запроса-параметра
указать модификатор ``\#'':
\query{@funcn(\#room)}


Если функция находится на не первом уровне иерархии, то запросы-параметры
будут выполняться относительно последнего элемента предыдущего уровня иерархии.
Например, в запросе 
\query{building (@count(room))}
Список комнат (\cl{room}), переданный в функцию \cl{@count} будет формироваться относительно
здания(\cl{building}), т.е. данный запрос выведет список зданий и количество комнат в них.

В запросе
\query{building (floor, room (@count(device)))}
функция \cl{@count} будет подсчитывать количество устройств(\cl{device}) относительно комнаты(\cl{room}), т.е.
запрос выведет количество устройств в комнате. 

Если необходимо, чтобы на не первом уровне иерархии запрос выполнялся независимо от предыдущего уровня,
то в начале запроса необходимо указать модификатор \cl{\%}:
\query{building (@funcn(\%room))}
В приведенном выше запросе в функцию \cl{@funcn} будет передан весь список объектов
\cl{room}.

%Будьте осторожны, используя одновременно два модификатора в запросах с вызовом функции и 
%списком связанных элементов в качестве параметра на не первом уровне иерархии кортежа.
%В запросе
%\query{building (@funcn(\%\#room))}
%для каждого объекта \cl{building} будет вызвана функцию \cl{@funcn} с каждым
%объектом \cl{room}.

Использование одновременно двух модификаторов запрещено.







\subsubsection{Рассогласованность по параметрам функции}
Запрос называется рассогласованным по параметрам функции, если в нем вызывается
функция, для которой невозможно определить с каким набором параметров следует вызывать данную функцию.
Рассмотрим пример:
\query{@funcn(\#building; \#room)}
В данном запросе функция вызывается с двумя параметрами, значением которых является
выполнение запросов
\query{building}
и
\query{room}
Однако, модификатор \cl{\#} говорит о том, что в функцию необходимо передавать не весь результат
запроса, а каждый кортеж в отдельности. Таким образом, в 
нашем примере непонятно с какими кортежами из результатов запросов-параметров нужно вызывать
функцию. Может показаться, что проблема решается путем простого вызова функции с кортежами из результатов 
запросов-параметров по порядку. Однако, непонятно как действовать в том случае, если 
количество этих кортежей различно. Поэтому ЯЗ не допускает рассогласованности по параметрам функции.

%Рассмотрим пример частично рассогласованного запроса по параметрам функции. В таком запросе
%существет возможность определить






\subsubsection{Стандартная библиотека функций}\label{std-funcs}
Реализация данной спецификации может предоставить любой набор функции для работы с 
данными запроса. Однако, этот набор должен содержать, по крайней мере, следующие
функции:
\begin{description}
    \item [avg] --- вычисляет среднее значение элементов списка-аргумента.\\
	{\it Входные данные:} набор кортежей, состоящих из одного элемента, над которым
			    реализация позволяет производить арифметические операции.\\
	{\it Результат:} значение выражения $$\frac{\sum_{i=1}^n e_i}{n}$$, 
		    где $e_i$ --- значение элемента $i$-го кортежа, 
		    $n$ --- количество кортежей.
    \item [count] --- подсчитывает количество элементов в списке.\\
	{\it Входные данные:} список кортежей.\\
	{\it Результат:} количество элементов в списке-аргументе.
    \item [first] --- возвращает первый кортеж из списка-аргумента.\\
	{\it Входные данные:} список кортежей.\\
	{\it Результат:} первый кортеж из списка-аргумента.
    \item [last] --- возвращает последний кортеж из списка-аргмента.\\
	{\it Входные данные:} список кортежей.\\
	{\it Результат:} последний кортеж из списка-аргумента.
%    \item [distinct] возвращает тольк\\
%	{\it Входные данные:} \\
%	{\it Результат:}
    \item [max] --- возвращает максимальное значение из элементов списка-аргумента.\\
	{\it Входные данные:} набор кортежей, состоящих из одного элемента, над которым
			    реализация позволяет производить операцию сравнения.\\
	{\it Результат:} максимальное значение из элементов списка-аргумента.
    \item [min] --- возвращает минимальное значение из элементов списка-аргумента.\\
	{\it Входные данные:} набор кортежей, состоящих из одного элемента, над которым
			    реализация позволяет производить операцию сравнения.\\
	{\it Результат:} минимальное значение из элементов списка-аргумента.
    \item [sum] --- возвращает сумму значений элементов списка-аргумента.\\
	{\it Входные данные:} набор кортежей, состоящих из одного элемента, над которым
			    реализация позволяет производить арифметическую операцию сложения.\\
	{\it Результат:} сумма значений элементов списка-аргумента.
\end{description}
