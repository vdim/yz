В данном разделе будут даны разъяснения относительно
понятий, употребляемых по тексту далее.

\begin{description}
    \item [Пользователь.] Нечто или некто, использующий язык ЯЗ.
    \item [Элемент результата запроса.] Элементом результата запроса
может быть: 
	\begin{itemize}\addtolength{\itemsep}{-0.3\baselineskip}
	    \item объект; 
	    \item значение атомарного типа (строка, число и т.д.);
	    \item список объектов или значений атомарного типа;
	    \item свойство объекта;
	    \item результат выполнения функции;
	    \item результат выполнения метода, вызванного у объекта или 
		свойства объекта (в том случае, если свойством объекта является объект).
	\end{itemize}
	Три последних пункта могут представлять собой объект или являться значением атомарного типа.
    \item [Тип элемента.] Тип элемента определяется элементом:
если элементом является объект, то типом элемента будет тип объекта, если
элементом является свойство объекта, то типом элемента будет тип свойства объекта и т.д. 
%если --- результат выполнения функции, то --- типом результата выполнения функции.
    \item [Кортеж.] Элемент со связанным с этим элементом списком кортежей. Как структура данных
реализация кортежа может быть записана на псевдокоде следующим образом:
\begin{verbatim}
    class Tuple {
        Object element;
        List<Tuple> tuples;
    }
\end{verbatim}
В графическом виде кортеж можно представить следующим образом:
\begin{center}
    \begin{tabular}{|l|l|l|}
	\hline
	element-1 & element-1-1 & element-1-1-1\\
	\cline{3-3}
				    & & element-1-1-2\\
	\cline{3-3}
				    & & element-1-1-3\\
	\cline{2-3}
		    & element-1-2 & element-1-2-1\\
	\cline{3-3}
				    & & element-1-2-2\\
	\hline
    \end{tabular}
\end{center}
В вышеприведенном примере наполнение структуры данных кортежа будет следующим:
\begin{verbatim}
[
  element = element-1
  tuples = [ 
             element = element-1-1
             tuples = [
                        element = element-1-1-1
                        tuples = [], 
                        element = element-1-1-2
                        tuples = [], 
                        element = element-1-1-3
                        tuples = []
             ],
             element = element-1-2
             tuples = [
                        element = element-1-2-1
                        tuples = [], 
                        element = element-1-2-2
                        tuples = []
             ]
  ]
]
\end{verbatim}


    \item [Иерархия кортежа.] Структура вложенности кортежа.
    \item [Структура кортежа.] Набор типов элементов, входящих в кортеж.
    \item [Результат запроса.] Набор кортежей или элемент.
    \item [Структура результата запроса.] Объединение структур кортежей. На основе заданной
пользователем структуры результата запроса будет произведена соответствующая выборка.
    \item [Рекурсивная ссылка.] Свойство класса, типом которого является этот же класс.
%Стоит отметить, что на набор структур кортежей в структуре результата запроса не накладывается
%ограничение на типовую и количественную однородность, таким образом результат запроса может объединять
%разные структуры кортежей. 

\end{description}


